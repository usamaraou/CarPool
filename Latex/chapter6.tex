\chapter{System Testing and Evaluation}\label{chap:testingEvaluation}

%\version{v1.11.2015}

\section*{}
System testing of software or hardware is testing conducted on a complete, integrated system to evaluate the system's compliance with its specified requirements. Be warned that many projects fall down through poor evaluation. Simply building a system and documenting its design and functionality is not enough to gain top marks. It is extremely important that you evaluate what you have done both in absolute terms and in comparison with existing techniques, software, hardware etc. This might involve quantitative evaluation and qualitative evaluation such as expressibility, functionality, ease-of-use etc. At some point you should also evaluate the strengths and weaknesses of what you have done. Avoid statements like "The project has been a complete success and we have solved all the problems associated with ...! It is important to understand that there is no such thing as a perfect project. Even the very best pieces of work have their limitations and you are expected to provide a proper critical appraisal of what you have done. The following are different types of testing that should be considered during System testing:

\begin{itemize}
	\item Graphical user interface testing
	\item Usability testing
	\item Software performance testing
	\item Compatibility testing
	\item Exception handling
	\item Load testing
	\item Security testing
	\item Installation testing
\end{itemize}

For research based projects this chapter should include complete description of evaluation metrics and analysis/discussion of evaluation results.
