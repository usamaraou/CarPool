\chapter{Conclusions and Future Work}\label{chap:conclusions}

The implementation from a concept to a real application was extremely challenging but it was very effective and full of experience. The final application turned out to be very professional and organized. It contained all necessary features to make both the driver and the passenger feel comfortable using it. From design to backend features, everything is checked. After a lot of testing and bug fixes the application finally reached a final state with no known bugs.\\

We have implemented all functional and non-functional needs and some of the optional needs. Some additional features were added to make the application more usable but the core concept stayed the same. Since it offers all the mentioned features and needs in the project description. It also respects the optimization needs such as good code design style, small sized application(4mb) and targeting as many android devices as possible.\\

By using Java which is the official programming language for android, we have made sure that the code was well structured as well as optimized. Since android itself is built on Java, there are plenty of Java libraries available and Java has a wide open-source ecosystem. Java apps are lighter even when compared to Kotlin apps, resulting in a faster app experience. Java yields a faster build process too, letting you code more in less time. Thanks to the accelerated assembly with gradle due to which assembling large projects becomes easier in Java.\\

As for the backend is concerned, our best choice was Firebase because it’s very powerful and simple to use, especially for beginners. It also contains a free usage tier with good performance. The application does its best to optimize the backend for both database usage and user experience. Firebase provided all necessary backend functions which helped making a professional application on time.\\
 
By using all the tools and libraries available and all knowledge of data structures and object-oriented programming, we were able to make a well-designed application that could be used without bugs and didn’t lack any important features. This implementation was definitely helpful to our career, since implementing a concept into a real application gives us good knowledge of the development environment.\\

Now that the application has been developed, we can look back and tell how much we have learned from this project. So many things could have been done differently, in terms of implementation and design concept also. Many additional features that an application like this could have to be more useful and competitive.\\

When the application was first being designed there were so many concerns that were not considered. By the time of developing and testing the application, we have realized so many things in terms of design and code optimization. Because of this, we modified so many things since we started working on the implementation just to make sure that application is more efficient and optimized. This proves that working on this project made us learn so many things about software development in general. Not only this, we also learned how to make better conceptual design and learned how to turn an idea into a real application.\\

There are some ways in which this application can be improved further. Some of the features for the application which are not yet present, but could be implemented in future development.\\

Our group has primarily been focusing on the functionality and simplicity of this application. If time allows, we we'll try to make the GUI more colourful and elegant, but there'll always be room for betterment in this segment.