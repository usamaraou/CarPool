\chapter{Introduction} \label{chap:intro}

\section{Project Background/Review}

\justify
Carpooling (also car-sharing, ride-sharing and lift-sharing) is sharing of car journeys so that more than one person travels in a car, and fulfills the need of others instead of driving to a location themselves. Drivers and passengers find a partner for journeys through multiple platforms available. After finding a match, they can communicate with each other to rearrange any details for the journey. Cost, meeting points and other details like space for luggage are agreed on. Then they meet and do their shared car journey as planned.\\

Ride sharing usually means to divide the travel expenses equally between all the occupants of the vehicle (driver or riders). Vehicle owner doesn’t want to earn money, but to cut down their fuel bills with the occupants of vehicle. The expenses divided include the fuel bills and tolls taxes. Multiple platforms are available that provide the facility of carpooling. Usually there’s a fare founded by the car driver and accepted by passengers sicne it is important they get an agreement before trip start.

By using the maximum capacity of a vehicle, ride sharing reduces each person’s travel costs such as fuel costs, tolls taxes, and also the stress of driving. Carpooling is environmental friendly and better way to travel as it helps reduces pollution, carbon dioxide emission,  traffc on the roads, and need of parking spaces. Ride sharing is a great way to fritter away the total spaciousness of a car, which might otherwise remain unused if it were just the one person travelling in the car.\\

Carpooling or ride sharing is more popular among the folks who work in same place or nearby places, and who board city area. However, carpooling is less likely among people those spend longer at work, elderly people, and homeowners.\\

\section{Problem description}
Many vehicle-owning students who commute on daily basis often have unoccupied seats in their vehicles. Many non-vehicle owning students find it very difficult sometimes to find ride for travelling to and from university. This project aims to provide a ride sharing application where non vehile owning students can share a ride with vehicle owning students.

\section{Project Objectives}
\textbf{Objective}
\begin{itemize}

\item To Allow vehicle owning students to share their rides with other students for traveling to and from their institutes and cut down their fuel bills.

\item To facilitate non-vehicle owning students for travelling to and from university easily and cheaply.

\end{itemize}
\textbf{Goals}
\begin{itemize}

\item Cost Effective: Much Cheaper than cab services.
\item Ease of getting ride: Riders are easy approachable, which reduces the tension of finding and catching of local transport right on time.
\item Fewer cars on the road will have reduced fuel consumption which will make environment eco-friendly.
\end{itemize}

\section{Project Scope}
This project (CarPool) aims to develop an android based application for carpooling for students. The application allows vehicle owning students to share rides with non-vehicle owning students and allows passengers to search for a ride while being secure.\\

This application will help students save money and also reduce the pollution of the environment. This application focuses on serving needs of students. CarPool is intended for the students of Air University and it supports android phones and tablets. Users will need internet connection to use the application to offer or find a common route to travel.\\

The application has a simple and easy interface. Users must register at first before using the application and after that they must choose between a driver or a rider. A driver can select riders while a rider request a ride to a location.

\section{The Degree of Project Report}
In our FYP-1, we presented our idea that how CarPool would be beneficial. The only purpose of FYP-1 was to present and defend the idea. We have completed both tasks successfully and we also developed some mockup screens to present our idea.\\

However, in FYP-2, the task assigned to us was to develop a working application for two users: driver and rider along with the implementation of the core feature of our application, which was location tracking of driver and rider.
