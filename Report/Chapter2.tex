\chapter{Literature Review} \label{chap:literatureReview}

\justify

In 2009, carpooling represented 43.5\% of all trips within the U.S and 10\% of commute trips. In 2011, a corporation called Greenock created a campaign to encourage others to use this type of transportation so as to reduce the emission of carbon dioxide. Carpooling, or car sharing, is promoted by a national UK charity, carplus, whose mission is to market responsible car use so as to alleviate financial, environmental and social costs of motoring. This contributes to British government's initiative to scale back congestion and parking pressure and contribute to relieving the burden on the environment and to the reduction of traffic related air-pollution.\\

"Cabbing All the Way” is a book written by author Jatin Kuberkar that narrates story of a carpool with twelve people on board. Based in the city of Hyderabad, India, the book is on a real-life narration and highlights the potential benefits of a carpool.

Many carpoolings applications and websites have been developed around the world. A similar carpooling system was developed in Massey University New Zealand by a group of students to allow students of Massey University, Albany Campus to share their vehicle with non-vehicle owning students. Following are some examples of carpooling systems around the globe: 

\section{Websites}
\begin{itemize}
	
	\item New Zealand: https://www.asa.ac.nz/carpool
	\item Algeria: www.nroho.com, www.m3aya.com, www.nsogo.net
	\item Europe: BlaBlaCar.com, carpooling.com, GoMore.com
	\item France: covoiturage.fr
	\item USA: car.ma, www.rdvouz.com
	\item World: Outpost.travel, joinntravel.com, www.letsride.in
	
\end{itemize}

\section{Mobile Applications}
\begin{itemize}
	
	\item New Zealand: ASA
	\item Algeria: YAssir, Nsogo, AMIR
	\item World: Uber, sRide, RideShare, 
	\item USA: Uber, Lyft
	\item France: Karos, Wever, BlaBlaCar, OuiHop

\end{itemize}

\section{Carpooling Types}
\subsection{Regular}
In this type of carpooling, the driver sits in the car in a closed space and (s)he is free to do what (s)he likes such as listen to the songs or call with headsets etc. In the United States, an intermediate concept has developed between carpooling and the public transport line, called the vanpool. These are minibuses chartered by an employer, a public authority or a private company and availabe to a group of people who regularly travel to the same route.

\subsection{Occasional}
This type of carpooling is especially used for freedom, peace, pleasure or last minute departures. The linking is usually done through websites or mobile applications, which may significantly reduce travel costs, but usually requires to carpool with one or more unknown persons. This type of carpooling is mainly used for freedom or last minute departures. The linking is commonly done through websites or mobile applications, which might significantly reduce travel costs, but usually requires to carpool with one or more unknown persons.

\subsection{Eventual}
Participants in an event (music festival, sporting event, wedding, associative or institutional meeting) can organize ride sharing to the venue of the event. This one-time carpool includes a special feature: all participants commute on the same route and to the same place on the same date. The type of carpooling is also used for departures on holidays or weekends.\\

There are also "cultural" carpooling platforms which provide facility to go to a cultural site: castles, museums, exhibitions, religious places, festivals, etc.

\section{Advantages and Disadvantages of Carpooling Applications}

\subsection{Advantages of carpooling}

\begin{itemize}
\item Cost saving: A ride may be twice as cheap than traveling in own vehicle or in a private cab.
\item A fewer number of cars on the road can reduce the emission of carbon dioxide and make the air cleaner. 
\item Saving time: Fewer cars - fewer traffic jams. It becomes possible to reach at destination faster and solve the issue to find a parking place.
\item Meeting new people. Traveling together allows you to find out good friends. 
\end{itemize}

\subsection{Disadvantages of carpooling}

\begin{itemize}
\item Indecent riders: Some people can attemp to discount the price or even ask the driver to pickup or to go to the place that's not on the scheduled route. And it is important to point out to  riders at the very beginning of trip.  
\item Indecent drivers: Unfortunately, drivers may overload the car and ask for a high price. 
\item In some cases, a driver has to pick up the each passenger separately. It increases the time of traveling.
\item Passengers may be not satisfied with the driving. In turn, the drivers are often annoyed with an excessive volubility of companions, their untidiness or lack of manners.
\end{itemize}