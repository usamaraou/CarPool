\chapter{Literature Review} \label{chap:literatureReview}

\section*{Carpooling}

\section{Definition and general principle}
\justify
Carpooling (also car-sharing, ride-sharing and lift-sharing) is that the sharing of car journeys in order to use the full capacity of a car, and full fill the need of others instead of drive to a location themselves.\\

By using the maximum capacity of a vehicle, ride sharing reduces each person’s travel costs such as fuel costs, tolls taxes, and also the stress of driving. Carpooling is environmental friendly and better way to travel by sharing one vehicle reduces pollution, carbon dioxide emission, reduce traffic on the roads, and also reduce the need of parking spaces. Officials often support carpooling, mostly during times of high pollution or high fuel prices. Ride sharing is a great way to fritter away the total spaciousness of a car, which might otherwise remain unused if it were just the one person travelling in the car.\\

In 2009, carpooling represented 43.5\% of all trips within the U.S and 10\% of commute trips. The bulk of carpool commutes (over 60\%) are ”fam-pools” with members of the family.\\

Carpooling or ride sharing is more popular among the folks who work in same place or nearby places, and who board city area. However, carpooling is less likely among people those spend longer at work, elderly people, and homeowners.\\

Ride sharing usually means to divide the travel expenses equally between all the occupants of the vehicle (driver or riders). Vehicle owner doesn’t want to earn money, but to cut down their fuel bills with the occupants of vehicle. The expenses divide include the fuel bills and tolls taxes.Multiple platforms available that provide the facility of carpooling. Usually there’s a fare founded by the car driver and accepted by passengers because they get an agreement before trip start.

\section{Carpooling Types}
\subsection{Regular}
The car is commonly refered as an extension of the private space, the driver, alone in a vehicle is in a closed space, he is free to do what he likes, listen to the song, sing, call with headsets etc. In the United States an intermediate concept has developed between carpooling and the public transport line, the vanpool. These are minibuses chartered by an employer, a public authority or a private company and availabe to a group of people who regularly travel to the same route.

\subsection{Occasional}
This type of carpooling is especially used for freedom, peace, pleasure or last minute departures. The linking is usually done through websites or mobile applications, which may significantly reduce travel costs, but usually requires to carpool with one or more unknown persons. This type of carpooling is mainly used for freedom or last minute departures. The linking is commonly done through websites or mobile applications, which might significantly reduce travel costs, but usually requires to carpool with one or more unknown. persons.

\subsection{Eventual}
Participants in an event (music festival, sporting event, wedding, associative or institutional meeting) can organize ride sharing to the venue of the event. This one-time carpool includes a special feature: all participants commute on the same route and to the same place on the same date. Carpooling is also used for departures on holidays or weekends, savings during travelling being even larger than the trip is long. So carpooling becomes an alternate of affordable and accessible transportation.\\

There are also ”cultural” carpooling platforms provide facility to go to a cultural site: castles, museums, exhibitions, artists’ studios, religious places, festivals, etc.
\section{Advantages and disadvantages of carpooling applications}
\subsection{Advantages of carpooling}
\begin{itemize}
\item The advantage of carpooling is cost saving. A ride may be twice as cheap than traveling in own vehicle or in a private cab.
\item The car isn't a gas cost only. There are some cost items for the upkeep, repair, parts replacement in your vehicle. If you reduce the usage of car utilization, you reduce these costs.
\item A fewer number of cars on the road can reduce the emission of carbon dioxide and make the air cleaner. 
\item Saving time. Fewer cars - fewer traffic jams. that is to say, it is possible to reach at destination faster and solve the issue to find a parking place.
\item Meeting new people. Traveling together allows you to find out good friends. 
\end{itemize}

\subsection{Disadvantages of carpooling}
\begin{itemize}
\item Indecent riders. Some people can attemp to discount the price or even ask the driver to pickup or to go to the place that's not on the scheduled route. And it is important to point out such riders at the very beginning of trip.  
\item Indecent drivers. Unfortunately, some drivers can play an unfair game similarly as an example, drivers who overload the car and arouse a high price. 
\item In some cases, a driver have to pick up the each passenger separately. It increases the time of traveling.
\item Passengers may be not satisfied with the driving. In turn, the drivers are often annoyed with an excessive volubility of companions, their untidiness or lack of manners.
\end{itemize}