\chapter{Introduction} \label{chap:intro}

%\version{v1.11.2015}

\section*{}
\section*{What is CarPool?}
\section{Project Background/ Review}
Carpooling (also car-sharing, ride-sharing and lift-sharing) is the sharing of car journeys so that more than one person travels in a car, and prevents the need for others to have to drive to a location themselves.
\\ Drivers and passengers offer and search for journeys through one of the several mediums available. After finding a match they contact each other to arrange any details for the journey(s). Costs, meeting points and other details like space for luggage are agreed on. They then meet and carry out their shared car journey(s) as planned.
\\ By having more people using one vehicle, carpooling reduces each person's travel costs such fuel costs, tolls and the stress of driving. Authorities often encourage carpooling, especially during periods of high pollution or high fuel prices. Car sharing is a good way to use up the full seating capacity of a car, which would otherwise remain unused if it were just the driver using the car.
\\ In 2009, carpooling represented 43.5% of all trips in the United States and 10% of commute trips. The majority of carpool commutes (over 60%) are "fam-pools" with family members.
In 2011, an organization called Greenock created a campaign to encourage others to use this form of transportation in order to reduce their own carbon footprint.
\\ Carpooling, or car sharing as it is called in British English, is promoted by a national UK charity, Carplus, whose mission is to promote responsible car use in order to alleviate financial, environmental and social costs of motoring today, and encourage new approaches to car dependency in the UK. Carplus is supported by transport for London, the British government initiative to reduce congestion and parking pressure and contribute to relieving the burden on the environment and to the reduction of traffic-related air-pollution.
\\ Cabbing All the Way is a book written by author Jatin Kuberkar that narrates a success story of a carpool with twelve people on board. Based in the city of Hyderabad, India, the book is a real-life narration and highlights the potential benefits of having a carpool

\section{Problem description}
Many vehicle-owning Students who commute on daily basis often have unoccupied seats in their vehicles. Many non-vehicle owning students find it very difficult sometimes to find ride for travelling to and from university.

\section{Project Objectives}
\textbf{Objective}
\begin{itemize}

\item To Allow vehicle owning students to share their rides with other students for traveling to and from their institutes and cut down their fuel bills.

\item To facilitate non-vehicle owning students for travelling to and from university easier and cheaper.

\end{itemize}
\textbf{Goals}
\begin{itemize}

\item Cost Effective: Much Cheaper than Cab services.
\item Ease of getting ride: Riders are easy approachable, which reduces the tension of finding and catching of local transport right on time.
\item Fewer Cars on the road will have reduced fuel consumption which will make environment Eco-friendly.
\end{itemize}

\section{Project Scope}
This project (CarPool) aims to develop an Android based application for carpooling 
for students, this application allows vehicle owning students to submit rides for specific 
targets and allows passengers to reserve/request rides from drivers all while being secure 
and having a simple interface.
\\ This application will help students save money and also reduce the pollution of the 
environment and effects of vehicles, this application focuses on serving needs of students.
CarPool will be intended for the students in Air University and it will support Android phones 
and Tablets, Users will need internet connection to use the application to offer or find a 
common route to travel.
\\ The application will have a simple and easy interface, Users must register at first before 
using the application, after that they must choose between a driver or a passenger, a driver 
can offer a drive to a specific location while a passenger can find or request a ride to a location.

\section{The Degree of Project Report}
In our FYP-1, we presented our idea that how CarPool would be beneficial. The only purpose of FYP-1 was to present and defend the idea. We have completed both tasks successfully and we also developed some mockup screens to present our idea.
\\ However, in fyp-2, the task assigned to us was to develop a working application for two users: driver and rider along with the implementation of the core feature of our application, which was location tracking of driver and rider, fetching current location, use Firebase it’s Real-time Database which is a NoSQL Fast Databaseand displaying that location on the map using Google Map API.
