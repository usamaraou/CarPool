\chapter{Conclusions}\label{chap:conclusions}
%\version{v1.10.2015}

\section*{}
\section{Conclusion}
The implementation from a concept to a real application was extremely challenging but it was very effective and full of experience, the final application turned out to be very professional and organized, it contained all necessary features to make both the Driver and the Passenger feel comfortable using it from design to backend features everything is checked. After a lot of testing and bug fixes the application finally reached a final state with no known bugs. 
\\  We have implemented all functional needs and non-functional needs and some of the optional needs, some additional features were added to make the application more usable but the core concept stayed the same since it offers all the mentioned features and needs in the project description it also respects the optimization needs such as good code design style, small application size (9mb) and targeting as many Android Devices as possible.
\\ By using Java which is the official programming language for Android we have made sure that the code was well structured as well as optimized, Since Android itself is built on Java, there are plenty of Java libraries to your aid. Also, Java has a wide open-source ecosystem.Java apps are lighter and more compact, even when compared to Kotlin apps, resulting in a faster app experience.Java yields a faster build process too, letting you code more in less time.Thanks to the accelerated assembly with Gradle, assembling large projects becomes easier in Java.
\\ As for the backend my best choice was Firebase because it’s very powerful and simple to use especially for beginners, it also contains a free usage tier with good performance. The application does its best to optimize the backend for both database usage and user experience. Firebase provided all necessary backend functions which made making a professional application in time possible. 
\\  By using all the tools and libraries available and by using all knowledge of Data Structures and Object-oriented programming we were able to make a well-designed application that could be used with no problems and didn’t lack any important features, this implementation was definitely helpful to my career since implementing a concept into a real Application gives good knowledge of the development environment. 

\begin{center}
    \section*{\huge{Final Conclusion}}
\end{center}
By the time the application was finished we can look back and tell how much we have learned from this project. So many things that should have been done differently in terms of implementation and design concept, so many additional features that an application like this has to have to be useful and competitive. 
\\ When the application was first being designed there were so many concerns that were not considered and by the time the testing happened and after all that time developing it we have realized that we learned so many things in terms of concept design and code optimizations and because of that we modified so many things since we started working on the implementation just to make the application more efficient and optimized. This proves that working on this project made us learn so many things about software development in general, not only that but we also learned how to make better conceptual design and learned how to turn an idea into a real application.
\\ Building the application from the ground up was a great experience, the most important thing that we learned and after doing so much work is, in this type of software/product it is difficult  to reach a final state so it is very important to deliver and get a first raw version and then fast and optimize and develop side features later. Some decisions were even based on this premise, having in mind that some choices are for the short-medium term rather than the long term are just faster to deliver that way, but finally we have ended up with a satisfying application that checks the main qualities of an application, these qualities being, good design, secure and efficient backend, and finally a clean and optimized coding design style. 



