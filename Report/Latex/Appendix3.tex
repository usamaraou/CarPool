\chapter{Steeple Analysis} \label{ap:Appendix1}

\section*{Social}
\begin{itemize}
\item Increase social interaction and solidarity.
\item Meet new people during rides and make new friends.
\item Driving with people is better than driving alone since it involves less stress.
\end{itemize}

\section*{Technology}
\begin{itemize}
\item Smartphone penetration is increasing day after day.  
\item Use of technology to create matches between drivers and passengers.
\item The application is accessible from anywhere using a smartphone  real time communication between actors.
\end{itemize}

\section*{Environmental}
\begin{itemize}
\item Increase of high occupancy vehicles which will lead to a decrease of CO2 emissions.
\item Less cars on the roads that leads to safer roads and fluent traffic. 
\end{itemize}

\section*{Economical}
\begin{itemize}
\item Savings as the price of gas and highways is shared among the travelers in a context of an increasing gas price.
\end{itemize}

\section*{Political}
\begin{itemize}
\item Increase of support for initiatives that decreases greenhouse gas emission support from government thank to the benefits of carpooling.
\end{itemize}

\section*{Legal}
\begin{itemize}
\item Insurances of drivers and passengers: In case of accidents, if the owner of the car have insurance it covers any medical expense. Carpool more than other racial and ethnic groups.
\item Ride sharing serves an important role in enhancing mobility in low-income, hostel students who are more likely to be unable to afford personal automobiles.
\end{itemize}

\section*{Ethical}
\begin{itemize}
\item Client confidentiality should be kept: all information related to trips’ history should only be communicated to their respective use.
\end{itemize} 
  