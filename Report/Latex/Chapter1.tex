\chapter{Introduction} \label{chap:intro}

\section*{What is CarPool?}

\section{Project Background/Review}
\justify
Carpooling (also car-sharing, ride-sharing and lift-sharing) is that the sharing of car journeys so that more than one person travels in a car, and full fill the need of others others instead of drive to a location themselves.\\ 

Drivers and passengers find a partner for journeys through one in all multiple platforms available. After finding a match they can communicate with each other to rearrange any details for the journey(s). Cost, meeting points and other details like space for luggage are agreed on. Then they meet and do their shared car journey(s) as planned.\\

Share a ride with other people, reduces each person’s travel costs such as fuel costs, tolls taxes and also the stress of driving. Officials often support carpooling, mostly in the times of high pollution or high fuel prices. Ride sharing could be a great way to achieve the goal, to use the full seating capacity of a car, other wise which might remain unused if it were the only driver travelling in the car.\\

In 2009, carpooling represented 43.5\% of all trips within the U.S and 10\% of commute trips. the bulk of carpool commutes (over 60\%) are ”fam-pools” with relations. In 2011, a corporation called Greenock created a campaign to encourage others to use this type of transportation so as to reduce the emission of carbon dioxide.\\

Carpooling, or car sharing, is promoted by a national UK charity, carplus, whose mission is to market responsible car use so as to alleviate financial, environmental and social costs of motoring today, and encourage new approaches to car dependency within the United Kingdom. Carplus is supported by transport for London, British people government initiative to scale back congestion and parking pressure and contribute to relieving the burden on the environment and to the reduction of traffic related air-pollution.\\

"Cabbing All the Way”, it is a book written by author Jatin Kuberkar that narrates story of a carpool with twelve people on board. Based in the city of Hyderabad, India, the book is on a real-life narration and highlights the potential benefits of a carpool.

\section{Problem description}
Many vehicle-owning students who commute on daily basis often have unoccupied seats in their vehicles. Many non-vehicle owning students find it very difficult sometimes to find ride for travelling to and from university.

\section{Project Objectives}
\textbf{Objective}
\begin{itemize}

\item To Allow vehicle owning students to share their rides with other students for traveling to and from their institutes and cut down their fuel bills.

\item To facilitate non-vehicle owning students for travelling to and from university easier and cheaper.

\end{itemize}
\textbf{Goals}
\begin{itemize}

\item Cost Effective: Much Cheaper than cab services.
\item Ease of getting ride: Riders are easy approachable, which reduces the tension of finding and catching of local transport right on time.
\item Fewer cars on the road will have reduced fuel consumption which will make environment eco-friendly.
\end{itemize}

\section{Project Scope}
This project (CarPool) aims to develop an android based application for carpooling for students, this application allows vehicle owning students to share rides with non-vehicle owning students and allows passengers to search for a ride all while being secure and having a simple interface.\\

This application will help students save money and also reduce the pollution of the environment and effects of vehicles, this application focuses on serving needs of students. CarPool will be intended for the students in Air University and it will support android phones and tablets, users will need internet connection to use the application to offer or find a common route to travel.\\

The application will have a simple and easy interface, users must register at first before using the application, after that they must choose between a driver or a
rider, a driver can select riders while a rider request a ride to a location.

\section{The Degree of Project Report}
In our FYP-1, we presented our idea that how CarPool would be beneficial. The only purpose of FYP-1 was to present and defend the idea. We have completed both tasks successfully and we also developed some mockup screens to present our idea.\\

However, in FYP-2, the task assigned to us was to develop a working application for two users: driver and rider along with the implementation of the core feature of our application, which was location tracking of driver and rider, fetching current location, use Firebase it’s real-time database which is a NoSQL fast databaseand displaying that location on the map using Google Map API.
